\documentclass[aspectratio=169]{beamer}
\usepackage{graphicx} 


\title{Studieren im Digitalen Zeitalter \\\small Tools und Methoden\\ V0.1}
\author{Florian Rössing}
\date{\today}

\begin{document}

\maketitle

\section{Einführung}
\begin{frame}{Ich}
    Kurzer Lebenslauf und Motivation
\end{frame}

\begin{frame}{Worüber rede ich?}
\tableofcontents[sectionstyle=show,subsectionstyle=show/hide]
\end{frame}

\section{Anforderungen von Studenten}
\begin{frame}{Übersicht}
Was sind klassische Arbeiten eines Studenten?
    \begin{itemize}
        \item Mitschreiben
        \item Aufgaben bearbeiten
        \item Quellen sichten
        \item Präsentieren
        \item Kollaborieren        
    \end{itemize}
    $\Rightarrow$ Wissen ansammeln und Teilen
\end{frame}
\section{Kriterien für die Auswahl von Tools}
\begin{frame}{Die FAIR Prinzipien}
    \begin{itemize}[]
        \item[] \textbf{F}indable - Auffindbar
        \begin{itemize}
            \item Eure Daten sollten leicht zu durchsuchen sein
            \item Eure Daten sollten gut strukturiert sein
        \end{itemize}
        \item[] \textbf{A}ccessible - Zugänglich
        \begin{itemize}
            \item Ihr solltet jederzeit Zugang zu euren Daten haben
            \begin{itemize}
                \item Habt immer eine lokale Kopie!
            \end{itemize}
        \item Macht eure Notizen auch euren Mitstreitern zugänglich
    \end{itemize}
        \item[] \textbf{I}nteroperable - Interoperabel
        \begin{itemize}
            \item Verwendet gängige Dateiformate
            \item Stellt eure Quellen offen, nicht nur PDFs
        \end{itemize}
        \item[] \textbf{R}eusable - Wiederverwendbar
        \begin{itemize}
            \item ? 
        \end{itemize}
    \end{itemize}
\end{frame}

\section{Ausgewählte Tools}

\subsection{Mitschriften}

\begin{frame}{Legt euch ein System für Notizen zu}
    \begin{block}
           
    \end{block}    
    \begin{itemize}
        \item OneNote
        \item Google Keep \& Google Docs
        \item Evernote
        \item \LaTeX - Overleaf
        \item Markdown Editoren    
    \end{itemize}
\end{frame}
\begin{frame}{Welche Features sind hilfreich?}
    \begin{enumerate}
        \item Notizen Verknüpfen
        \item Verknüpfungen zu Quellen herstellen
        
    \end{enumerate}
    
\end{frame}

\subsection{Kollaboration}

\begin{frame}{Daten Teilen}
    \begin{itemize}
        \item Cloud Dienste
            \begin{itemize}
                \item Google Drive
                \item OneDrive
                \item Dropbox
                \item Sciebo
                \begin{itemize}
                    \item Eine offene Cloud die von deutschen Hochschulen bereitgestellt wird. Studenten bekommen 30 GB kostenlosen Speicherplatz.
                \end{itemize}
            \end{itemize}
        \item GitHub 
        \begin{itemize}
            \item Git ist ein Versionskontroll-Tool aus der Softwareentwicklung. Auf GitHub kann man seine Projekte veröffentlichen.
        \end{itemize}
    \end{itemize}    
\end{frame}

\begin{frame}{Collaboration}
    \begin{itemize}
        \item WhatsApp
        \item Slack
        \item Discord
        \item Matrix        
    \end{itemize}
\end{frame}

\subsection{Literatur Recherche}
\begin{frame}{Recherche Tools}
    \begin{itemize}
        \item Google, Google Scholar
        \item Connected Papers
        \item ChatGPT
        \item Eure Uni Bibliothek!
    \end{itemize}    
\end{frame}

\begin{frame}{Literatur Verwaltung}
    \begin{itemize}
        \item Zotero
        \item Citavi
        \item Mendeley
        \item ...
    \end{itemize}    
\end{frame}

\begin{frame}{Lernt Googlen}
    \begin{itemize}
        \item \textit{site:} beschränke die Suche auf eine Webseite
        \item \textit{filetype:} sucht nach Dateien vom Typ
        \item \textit{""}  Muss im Ergebnis vorkommen
        \item \textit{-} Darf nicht im Ergebnis vorkommen
        \item \textit{AND,OR} Logisches Verknüpfen von Suchbegriffen
    \end{itemize}
\end{frame}
\begin{frame}{Einer für Alle?}
    \begin{itemize}
        \item Google Workspace
        \item Office365
        \item Sciebo?
    \end{itemize}
    
    \begin{itemize}
        \item[+] bieten Tools für so ziemlich jeden Bedarf.
        \item[x] sperrt euch in dem jeweiligen Ökosystem ein
        \item[x] Möglicherweise durch Arbeitgeber gesperrt -> keine Zukunftsaussichten?
    \end{itemize}  
\end{frame}
\section{Addendum}
\subsection{Günstige Hardware}
\begin{frame}{Ein Laptop für den schmalen Euro}
    
\end{frame}
\subsection{Support}
\begin{frame}{Das HRZ}
    
\end{frame}
\begin{frame}{Online Communities}
    
\end{frame}

\end{document}
