\documentclass{beamer}
\usepackage{graphicx} 

\title{Studieren im Digitalen Zeitalter \\\small Tools und Methoden\\ V0.1}
\author{Florian Rössing}
\date{\today}

\begin{document}

\maketitle

\section{Einführung}
\begin{frame}{Ich}
    Kurzer Lebenslauf und Motivation
\end{frame}
\begin{frame}{Worüber rede ich?}
    \tableofcontents
\end{frame}

\section{Anforderungen von Studenten}
\begin{frame}{Übersicht}
Was sind klassische Arbeiten eines Studenten?
    \begin{itemize}
        \item Mitschreiben
        \item Aufgaben bearbeiten
        \item Quellen sichten
        \item Kollaborieren
    \end{itemize}
    $\Rightarrow$ Wissen ansammeln und Teilen
\end{frame}
\section{Kriterien für die Auswahl von Tools}
\begin{frame}{Die FAIR Prinzipien}
    \begin{itemize}[]
        \item[] \textbf{F}indable - Auffindbar
        \item[] \textbf{A}ccessible - Zugänglich
        \item[] \textbf{I}nteroperable - Interoperabel
        \item[] \textbf{R}eusable - Wiederverwendbar
    \end{itemize}
\end{frame}

\begin{frame}{Findable - Auffindbar}
    \begin{itemize}
        \item Eure Daten sollten leicht zu durchsuchen sein
        \item Eure Daten sollten gut strukturiert sein
    \end{itemize}
\end{frame}

\begin{frame}{Accessible - Zugänglich}
    \begin{itemize}
        \item Ihr solltet jederzeit Zugang zu euren Daten haben
        \begin{itemize}
            \item 
        \end{itemize}
        
    \end{itemize}
\end{frame}

\begin{frame}{Interoperable - Interoperabel}
    \begin{itemize}
        \item
    \end{itemize}
\end{frame}

\begin{frame}{Reusable - Wiederverwendbar}
    \begin{itemize}
        \item
    \end{itemize}
\end{frame}
\section{Ausgewählte Tools}
\begin{frame}{Klassische Cloud Dienste}
    
\end{frame}

\section{Addendum}
\subsection{Günstige Hardware}
\begin{frame}{Ein Laptop für den schmalen Euro}
    
\end{frame}
\subsection{Support}
\begin{frame}{Das HRZ}
    
\end{frame}
\begin{frame}{Online Communities}
    
\end{frame}

\end{document}
